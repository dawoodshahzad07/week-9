\documentclass[12pt,a4paper]{report}

% Packages
\usepackage[utf8]{inputenc}
\usepackage[T1]{fontenc}
\usepackage{graphicx}
\usepackage{listings}
\usepackage{xcolor}
\usepackage{hyperref}
\usepackage{float}
\usepackage{geometry}
\usepackage{titlesec}
\usepackage{minted}
\usepackage{tikz}
\usepackage{pgf-umlcd}

% Page geometry
\geometry{
    a4paper,
    total={170mm,257mm},
    left=20mm,
    top=20mm,
}

% Colors
\definecolor{javared}{rgb}{0.6,0,0}
\definecolor{javagreen}{rgb}{0.25,0.5,0.35}
\definecolor{javapurple}{rgb}{0.5,0,0.35}
\definecolor{javacomment}{rgb}{0.5,0.5,0.5}

% Listing style
\lstset{
    language=Java,
    basicstyle=\ttfamily\small,
    keywordstyle=\color{javapurple}\bfseries,
    stringstyle=\color{javared},
    commentstyle=\color{javagreen},
    morecomment=[s][\color{javacomment}]{/**}{*/},
    numbers=left,
    numberstyle=\tiny\color{black},
    stepnumber=1,
    numbersep=10pt,
    tabsize=4,
    showspaces=false,
    showstringspaces=false,
    frame=single,
    breaklines=true
}

% Title formatting
\titleformat{\chapter}[display]
{\normalfont\huge\bfseries}{\chaptertitle}{20pt}{\Huge}

\begin{document}

\title{
    {\Huge Gym Database Management System (GDMS)}\\
    \large A Comprehensive Java-Based Solution
}
\author{Project Documentation}
\date{\today}

\maketitle
\tableofcontents

\chapter{Introduction}
\section{Project Overview}
The Gym Database Management System (GDMS) is a comprehensive software solution designed to manage gym operations efficiently. This system handles member management, equipment tracking, payment processing, and various other aspects of gym administration.

\section{Objectives}
\begin{itemize}
    \item Streamline gym membership management
    \item Automate payment processing and tracking
    \item Monitor gym equipment maintenance
    \item Provide secure authentication and authorization
    \item Generate reports and analytics
\end{itemize}

\chapter{System Architecture}
\section{Overview}
The system follows a layered architecture pattern with clear separation of concerns:

\begin{itemize}
    \item Client Layer
    \item Security Layer
    \item Business Logic Layer
    \item Domain Layer
    \item Data Access Layer
    \item Database Layer
\end{itemize}

\section{Technology Stack}
\begin{itemize}
    \item Java 17
    \item Spring Security
    \item MySQL Database
    \item Hibernate ORM
    \item JUnit 5
    \item Maven
\end{itemize}

\chapter{Implementation}
\section{Domain Models}

\subsection{Member Class}
\begin{lstlisting}[caption=Member.java]
package com.gdms.model;

import org.springframework.security.crypto.bcrypt.BCryptPasswordEncoder;
import java.time.LocalDate;

public class Member extends User {
    private String membershipType;
    private LocalDate membershipStartDate;
    private LocalDate membershipEndDate;
    private static final BCryptPasswordEncoder passwordEncoder = 
        new BCryptPasswordEncoder();

    public Member(String name, String email, String password, 
                 String membershipType) {
        super(name, email, passwordEncoder.encode(password));
        this.membershipType = membershipType;
        this.membershipStartDate = LocalDate.now();
        this.membershipEndDate = membershipStartDate.plusMonths(1);
    }

    public boolean isMembershipActive() {
        return LocalDate.now().isBefore(membershipEndDate);
    }

    public void renewMembership(int months) {
        if (membershipEndDate.isBefore(LocalDate.now())) {
            membershipStartDate = LocalDate.now();
        } else {
            membershipStartDate = membershipEndDate;
        }
        membershipEndDate = membershipStartDate.plusMonths(months);
    }

    @Override
    public String getRole() {
        return "MEMBER";
    }

    // Getters and setters
    public String getMembershipType() {
        return membershipType;
    }

    public void setMembershipType(String membershipType) {
        this.membershipType = membershipType;
    }

    public LocalDate getMembershipStartDate() {
        return membershipStartDate;
    }

    public LocalDate getMembershipEndDate() {
        return membershipEndDate;
    }
}
\end{lstlisting}

\subsection{Equipment Class}
\begin{lstlisting}[caption=Equipment.java]
package com.gdms.model;

import java.time.LocalDate;

public class Equipment {
    private int equipmentId;
    private String name;
    private String status;
    private String location;
    private String description;
    private LocalDate lastMaintenanceDate;
    private LocalDate nextMaintenanceDate;

    public Equipment(int equipmentId, String name, String location, 
                    String description) {
        this.equipmentId = equipmentId;
        this.name = name;
        this.status = "Available";
        this.location = location;
        this.description = description;
        this.lastMaintenanceDate = LocalDate.now();
        this.nextMaintenanceDate = 
            LocalDate.now().plusMonths(1);
    }

    public void performMaintenance() {
        this.lastMaintenanceDate = LocalDate.now();
        this.nextMaintenanceDate = 
            LocalDate.now().plusMonths(1);
        this.status = "Available";
    }

    public boolean isMaintenanceDue() {
        return LocalDate.now().isAfter(nextMaintenanceDate);
    }

    public void reportIssue(String issue) {
        this.status = "Maintenance Required: " + issue;
    }

    // Getters and setters
    public int getEquipmentId() {
        return equipmentId;
    }

    public String getName() {
        return name;
    }

    public String getStatus() {
        return status;
    }

    public void setStatus(String status) {
        this.status = status;
    }

    public String getLocation() {
        return location;
    }

    public void setLocation(String location) {
        this.location = location;
    }

    public String getDescription() {
        return description;
    }

    public LocalDate getLastMaintenanceDate() {
        return lastMaintenanceDate;
    }

    public LocalDate getNextMaintenanceDate() {
        return nextMaintenanceDate;
    }
}
\end{lstlisting}

\subsection{Payment Class}
\begin{lstlisting}[caption=Payment.java]
package com.gdms.model;

import java.time.LocalDateTime;

public class Payment {
    private int paymentId;
    private int memberId;
    private double amount;
    private String status;
    private String paymentMethod;
    private String transactionId;
    private LocalDateTime paymentDate;

    public Payment(int paymentId, int memberId, double amount, 
                  String paymentMethod) {
        this.paymentId = paymentId;
        this.memberId = memberId;
        this.amount = amount;
        this.paymentMethod = paymentMethod;
        this.status = "Pending";
        this.paymentDate = LocalDateTime.now();
    }

    public boolean processPayment(String transactionId) {
        this.transactionId = transactionId;
        this.status = "Completed";
        return true;
    }

    public boolean refundPayment(String reason) {
        if ("Completed".equals(this.status)) {
            this.status = "Refunded: " + reason;
            return true;
        }
        return false;
    }

    public String generateReceipt() {
        return String.format("""
            Receipt
            --------
            Payment ID: %d
            Member ID: %d
            Amount: $%.2f
            Status: %s
            Payment Method: %s
            Transaction ID: %s
            Date: %s
            """,
            paymentId, memberId, amount, status,
            paymentMethod, transactionId, paymentDate);
    }

    // Getters and setters
    public int getPaymentId() {
        return paymentId;
    }

    public int getMemberId() {
        return memberId;
    }

    public double getAmount() {
        return amount;
    }

    public String getStatus() {
        return status;
    }

    public String getPaymentMethod() {
        return paymentMethod;
    }

    public String getTransactionId() {
        return transactionId;
    }

    public LocalDateTime getPaymentDate() {
        return paymentDate;
    }
}
\end{lstlisting}

\chapter{Testing}
\section{Unit Tests}

\subsection{Member Tests}
\begin{lstlisting}[caption=MemberTest.java]
package com.gdms.model;

import org.junit.jupiter.api.BeforeEach;
import org.junit.jupiter.api.Test;
import java.time.LocalDate;
import static org.junit.jupiter.api.Assertions.*;

public class MemberTest {
    private Member member;

    @BeforeEach
    void setUp() {
        member = new Member("John Doe", "john@example.com", 
                          "password123", "Premium");
    }

    @Test
    void testMemberCreation() {
        assertNotNull(member);
        assertEquals("John Doe", member.getName());
        assertEquals("john@example.com", member.getEmail());
        assertEquals("Premium", member.getMembershipType());
        assertTrue(member.isMembershipActive());
    }

    @Test
    void testAuthentication() {
        assertTrue(member.authenticate("password123"));
        assertFalse(member.authenticate("wrongpassword"));
    }

    @Test
    void testMembershipActive() {
        assertTrue(member.isMembershipActive());
        // Set end date to yesterday
        member.setMembershipEndDate(LocalDate.now().minusDays(1));
        assertFalse(member.isMembershipActive());
    }

    @Test
    void testRenewMembership() {
        LocalDate originalEndDate = member.getMembershipEndDate();
        member.renewMembership(3);
        assertTrue(member.getMembershipEndDate().isAfter(originalEndDate));
        assertEquals(3, member.getMembershipEndDate().getMonthValue() - 
                       originalEndDate.getMonthValue());
    }

    @Test
    void testGetRole() {
        assertEquals("MEMBER", member.getRole());
    }
}
\end{lstlisting}

\subsection{Equipment Tests}
\begin{lstlisting}[caption=EquipmentTest.java]
package com.gdms.model;

import org.junit.jupiter.api.BeforeEach;
import org.junit.jupiter.api.Test;
import java.time.LocalDate;
import static org.junit.jupiter.api.Assertions.*;

public class EquipmentTest {
    private Equipment equipment;

    @BeforeEach
    void setUp() {
        equipment = new Equipment(1, "Treadmill", "Cardio Area", 
                                "Commercial Grade Treadmill");
    }

    @Test
    void testEquipmentCreation() {
        assertNotNull(equipment);
        assertEquals(1, equipment.getEquipmentId());
        assertEquals("Treadmill", equipment.getName());
        assertEquals("Available", equipment.getStatus());
        assertEquals("Cardio Area", equipment.getLocation());
    }

    @Test
    void testPerformMaintenance() {
        LocalDate oldMaintenanceDate = equipment.getLastMaintenanceDate();
        equipment.performMaintenance();
        assertTrue(equipment.getLastMaintenanceDate()
                          .isAfter(oldMaintenanceDate));
        assertEquals("Available", equipment.getStatus());
    }

    @Test
    void testIsMaintenanceDue() {
        assertFalse(equipment.isMaintenanceDue());
        // Set next maintenance date to yesterday
        equipment.setNextMaintenanceDate(LocalDate.now().minusDays(1));
        assertTrue(equipment.isMaintenanceDue());
    }

    @Test
    void testReportIssue() {
        equipment.reportIssue("Motor problem");
        assertEquals("Maintenance Required: Motor problem", 
                    equipment.getStatus());
    }
}
\end{lstlisting}

\subsection{Payment Tests}
\begin{lstlisting}[caption=PaymentTest.java]
package com.gdms.model;

import org.junit.jupiter.api.BeforeEach;
import org.junit.jupiter.api.Test;
import static org.junit.jupiter.api.Assertions.*;

public class PaymentTest {
    private Payment payment;

    @BeforeEach
    void setUp() {
        payment = new Payment(1, 1, 99.99, "Credit Card");
    }

    @Test
    void testPaymentCreation() {
        assertNotNull(payment);
        assertEquals(1, payment.getPaymentId());
        assertEquals(1, payment.getMemberId());
        assertEquals(99.99, payment.getAmount());
        assertEquals("Pending", payment.getStatus());
        assertEquals("Credit Card", payment.getPaymentMethod());
    }

    @Test
    void testProcessPayment() {
        assertTrue(payment.processPayment("TXN123"));
        assertEquals("Completed", payment.getStatus());
        assertEquals("TXN123", payment.getTransactionId());
    }

    @Test
    void testRefundPayment() {
        payment.processPayment("TXN123");
        assertTrue(payment.refundPayment("Customer request"));
        assertTrue(payment.getStatus()
                        .startsWith("Refunded: Customer request"));
    }

    @Test
    void testGenerateReceipt() {
        payment.processPayment("TXN123");
        String receipt = payment.generateReceipt();
        assertTrue(receipt.contains("Payment ID: 1"));
        assertTrue(receipt.contains("Amount: $99.99"));
        assertTrue(receipt.contains("Transaction ID: TXN123"));
    }
}
\end{lstlisting}

\chapter{Project Configuration}
\section{Maven Configuration}
\begin{lstlisting}[caption=pom.xml]
<?xml version="1.0" encoding="UTF-8"?>
<project xmlns="http://maven.apache.org/POM/4.0.0"
         xmlns:xsi="http://www.w3.org/2001/XMLSchema-instance"
         xsi:schemaLocation="http://maven.apache.org/POM/4.0.0 
                           http://maven.apache.org/xsd/maven-4.0.0.xsd">
    <modelVersion>4.0.0</modelVersion>

    <groupId>com.gdms</groupId>
    <artifactId>gym-management-system</artifactId>
    <version>1.0-SNAPSHOT</version>

    <properties>
        <maven.compiler.source>17</maven.compiler.source>
        <maven.compiler.target>17</maven.compiler.target>
        <project.build.sourceEncoding>UTF-8</project.build.sourceEncoding>
    </properties>

    <dependencies>
        <!-- JUnit Jupiter for testing -->
        <dependency>
            <groupId>org.junit.jupiter</groupId>
            <artifactId>junit-jupiter</artifactId>
            <version>5.9.2</version>
            <scope>test</scope>
        </dependency>

        <!-- MySQL Connector -->
        <dependency>
            <groupId>mysql</groupId>
            <artifactId>mysql-connector-java</artifactId>
            <version>8.0.33</version>
        </dependency>

        <!-- Hibernate ORM -->
        <dependency>
            <groupId>org.hibernate</groupId>
            <artifactId>hibernate-core</artifactId>
            <version>6.2.7.Final</version>
        </dependency>

        <!-- Spring Security -->
        <dependency>
            <groupId>org.springframework.security</groupId>
            <artifactId>spring-security-crypto</artifactId>
            <version>6.1.2</version>
        </dependency>

        <!-- SLF4J API -->
        <dependency>
            <groupId>org.slf4j</groupId>
            <artifactId>slf4j-api</artifactId>
            <version>2.0.7</version>
        </dependency>

        <!-- Logback Classic -->
        <dependency>
            <groupId>ch.qos.logback</groupId>
            <artifactId>logback-classic</artifactId>
            <version>1.4.8</version>
        </dependency>

        <!-- Apache Commons Logging -->
        <dependency>
            <groupId>commons-logging</groupId>
            <artifactId>commons-logging</artifactId>
            <version>1.2</version>
        </dependency>
    </dependencies>

    <build>
        <plugins>
            <plugin>
                <groupId>org.apache.maven.plugins</groupId>
                <artifactId>maven-compiler-plugin</artifactId>
                <version>3.11.0</version>
                <configuration>
                    <source>${maven.compiler.source}</source>
                    <target>${maven.compiler.target}</target>
                </configuration>
            </plugin>
            
            <plugin>
                <groupId>org.apache.maven.plugins</groupId>
                <artifactId>maven-surefire-plugin</artifactId>
                <version>3.0.0</version>
            </plugin>
        </plugins>
    </build>
</project>
\end{lstlisting}

\chapter{Future Enhancements}
\section{Planned Features}
\begin{itemize}
    \item Integration with fitness tracking devices
    \item Mobile application development
    \item Advanced analytics and reporting
    \item Class scheduling system
    \item Trainer management module
    \item Nutrition tracking system
\end{itemize}

\section{Technical Improvements}
\begin{itemize}
    \item Implement caching with Redis
    \item Add message queuing with RabbitMQ
    \item Enhance security features
    \item Implement real-time notifications
    \item Add backup and disaster recovery
\end{itemize}

\chapter{Conclusion}
The Gym Database Management System provides a robust solution for gym management needs. The system's modular architecture ensures scalability and maintainability, while comprehensive testing ensures reliability. Future enhancements will further improve the system's capabilities and user experience.

\end{document} 